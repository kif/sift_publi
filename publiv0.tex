\documentclass[preprint]{iucr}
 \papertype{CP}
 \journalcode{S}



\usepackage[T1]{fontenc}
\usepackage[utf8]{inputenc}

\begin{document}


\title{SIFT\_PyOCL : an implementation of SIFT in OpenCL}
\shorttitle{SIFT\_PyOCL}

    \author[a]{Jerome}{Kieffer}
    \author[b]{Pierre}{Paleo}
    %\author[c]{...}{...}
    \aff[a]{European Synchrotron Radiation Facility, \city{Grenoble}, \country{France}}
    %\aff[b]{}
    %\aff[c]{European Synchrotron Radiation Facility, \city{Grenoble}, \country{France}}
	\shortauthor{Kieffer et al.}

\maketitle

\begin{synopsis}
An OpenCL implementation of Scale-Invariant Feature Transform for image alignment
\end{synopsis}

\begin{abstract}
SIFT\_PyOCL is a Python module for fast image alignment, using the Scale-Invariant Feature Transform. The parallel implementation enables crucial speed-ups while being adaptable to various setups. The code can be executed on different platforms, including graphic cards, multi-core processors and accelerators.
 
\end{abstract}

\section{Introduction}

\section{Image alignment for synchrotron}
In various experiments, image alignment is required in data pre-processing.
\subsection{Coherent diffraction imaging}

\subsection{Full-Field XANES}
The European Synchrotron Radiation Facility (ESRF) beamline ID21 developed a full-field method for X-ray absorption near-edge spectroscopy\cite{fullfield} (XANES). %TODO : define 'flat' ?
%designed to perform submicronic XANES analysis in transmission mode on major elements of large samples. For each energy point across a given K- or L-edge, a magnified 2D transmission image of the sample is acquired by a camera coupled to an X-ray scintillator and magnifying visible light optics. Then, a ?flat field image? recorded with sample out of the x-ray beam is used for normalization. The vacuum-air interface is realised through a viewport inserted between the scintillator and the visible light objectives. A XANES stack consists of a series of normalized images that characterize the sample absorption across the absorption-edge of interest. The XANES spectra can then be extracted for each pixel of the image.
Since the flat field images are not acquired simultaneously with the sample transmission images, a realignment procedure has to be performed.

 


\section{Scale-Invariant Feature Transform}
%while phase correlation .... need a more robust algorithm
%SIFT can be adapted to various setups
%importance de la vitesse ... necessite de la parallelisation
%algo brevete aux US, mais libre utilisation pour la recherche
Phase correlation was used and efficient to correct translations, but had some failure cases. For example, when sticky tape was put on a sample, the alignment was done on the borders rather than on the sample. Therefore, a more robust method had to be used.

The Scale Invariant Feature Transform\cite{lowe2004} (SIFT) uses control points to align images, making it much more versatile than phase correlation. Currently used on the ESRF beamline ID21, it takes about 8 seconds per frame, and one stack can have up to 500 frames. Although the process is distributed on the EDNA\cite{edna} framework, the performance limits are reached. Image alignment is a crucial step in data pre-processing, hence the importance of accelerating it to ensure a fast data rate.

With the advent of high-performance GPU computing, many scientific data analysis programs were accelerated (\cite{pyfai}, \cite{Favre-Nicolin}). %TODO : more




Python \cite{python} is a scripting language that is very popular among scientists
and which also allows well structured applications and libraries to be developed.




\ack{Acknowledgements}




\bibliographystyle{iucr}
\bibliography{biblio}
%\referencelist[biblio]



\end{document}
