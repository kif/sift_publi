\documentclass[preprint]{iucr}
 \papertype{CP}
 \journalcode{S}



\usepackage[T1]{fontenc}
\usepackage[utf8]{inputenc}

\begin{document}


\title{SIFT\_PyOCL : an implementation of SIFT in OpenCL}
\shorttitle{SIFT\_PyOCL}

    \author[a]{Jerome}{Kieffer}
    \author[b]{Pierre}{Paleo}
    %\author[c]{Jonathan P.}{Wright}
    \author[c]{Ga\"el}{Goret}
    \cauthor[c]{J\'er\^ome}{Kieffer}{jerome.kieffer@esrf,fr}{}
    \aff[a]{European Synchrotron Radiation Facility, \city{Grenoble}, \country{France}}
    %\aff[b]{}
    %\aff[c]{European Synchrotron Radiation Facility, \city{Grenoble}, \country{France}}
	\shortauthor{Kieffer et al.}

\maketitle

\begin{synopsis}
An OpenCL implementation of Scale-Invariant Feature Transform for image alignment
\end{synopsis}

\begin{abstract}
SIFT\_PyOCL 


FabIO is a Python module written for easy and transparent reading 
of raw two-dimensional data from various X-ray detectors. The module provides a
function for reading any image and returning a fabioimage object which 
contains both metadata (header information) and the raw data.
All fabioimage object offer additional methods to extract
information about the image and to open other detector images from
the same data series.
 
\end{abstract}

\section{Introduction}


\section{FabIO Python module}

Python \cite{python} is a scripting language that is very popular among scientists
and which also allows well structured applications and libraries to be developed.




\ack{Acknowledgements}

We acknowledge Andy G\"otz and Kenneth Evans for extensive testing when including
the FabIO reader in the Fable ImageViewer \cite{fable}.
We also thank V. Armando Sol\'e for assistance with his TiffIO reader and
Carsten Gundlach for deployment of FabIO at the beamlines i711 and i811, 
MAX IV and providing bug reports.
We finally acknowledge our colleagues who have reported bugs and helped to
improve FabIO.
Financial support was granted by the EU 6th Framework NEST/ADVENTURE project
TotalCryst \cite{totalcryst}.



\bibliographystyle{iucr}
\bibliography{biblio}
%\referencelist[biblio]



\end{document}
