\documentclass[preprint]{iucr}
 \papertype{CP}
 \journalcode{S}



\usepackage[T1]{fontenc}
\usepackage[utf8]{inputenc}

\begin{document}


\title{SIFT\_PyOCL : an implementation of SIFT in OpenCL}
\shorttitle{SIFT\_PyOCL}

    \author[a]{Pierre}{Paleo}
    \author[a]{Emeline}{Pouyet}
    \cauthor[a]{J\'er\^ome}{Kieffer}{jerome.kieffer@esrf,fr}{}
    \aff[a]{European Synchrotron Radiation Facility, \city{Grenoble}, \country{France}}
    \shortauthor{Paleo et Kieffer}

\maketitle

\begin{synopsis}
An Python-OpenCL implementation of Scale-Invariant Feature Transform for image
alignment
\end{synopsis}

\begin{abstract}
SIFT\_PyOCL is a Python module implementing a parallel version of the Scale
Invariant Feature Transform algoritm in OpenCL, providing high-speed image
registration and alignment both on processors and on graphics cards. 
The sub-pixel precision measured is suitable for Full-Field X-ray
absorption spectroscopy stack alignement and the speed allows the processing of
such dataset online (i.e in real-time).
 
\end{abstract}

\section{Introduction}

Image alignment has been requested by many synchrotron beamlines for
various techniques like speckle images reconstruction in the field of coherent
X-ray diffraction imaging with module based pixel detectors or image stack
alignement in the field of  

 To address this need we developped a parallel
version of the SIFT algorithm for image registration working both on multi-core system and on graphics cards
and interfaced in Python, a free scripting language very popular among
scientists. This high performance computing library complements the toolbox
based on NumPy\cite{numpy}, SciPy\cite{scipy} and skimage\cite{skimage}. 


\section{Scientific applications for image alignment}

\subsection{Coherent X-Ray diffraction}

\subsection{Full-Field X-ray absorption spectroscopy} 


While phase correlation has intensively been used during the development of
FFXAS for image alignment, this algorithm is limited to translation and turned
out to be very sensitive to artifacts, among those: difference of intensity on
the image border, defects on the syntillator or on the camera\ldots 
Those defect could be corrected by some clever and sample specific
pre-processing (i.e. apodisation of the signal on the borders).


\section{Image alignemnt algorithm}

\subsection{The limits of phase correation} 

\subsection{Choice of the registration algorithm}

The SIFT algorithm \cite{Lowe1999,Lowe2004} which has been developped
for image registration and widely used for panoramic image stitching has beed
adapted from the IPOL\cite{ASIFT} implementation. Image alignment obtained in
FFXAS were similar in qualty to phase corelation with a better robustness to
artefacts. Moreover keypoints in some defective region of the image can be
discarded easily using a mask.


\subsection{how it works}


\section{Implementation}
While the phase correlation algorithm has been easily ported on graphics card
thanks to PyCUDA and cuFFT, hence very fast for online data
pre-processing; 
the SIFT algorithm is much more complicated and the implementation available was
single threaded.

\subsection{Parallelization of the algorithm}

\subsection{Instalation and usage}
\textem{Sift_pyocl} can, as any Python module, be installed from its sources,
available on github:
https://github.com/kif/sift_pyocl/archive/master.zip 
While \textem{Sift_pyocl} is open source and licensed under a very
permissive BSD license; one should notice the SIFT algorithm itself is
patented by the Univerity of Columbia\cite{SIFT}; neverthless this patent does
not apply in Europe.

Beside Python (version 2.6 or 2.7) and NumPy, \textem{Sift_pyocl} needs
PyOpenCL\cite{pyopencl}.
It was tested with the OpenCL\cite{opencl} driver from Nvidia on a
large variete of their GPUs and on multi-core processors with the driver from
Intel and AMD. The full installation procedure with testing is simply:
\textem{python setup.py build test install}


\subsection{Examples}

In this section we have collected some basic examples of how
\textem{sift_pyocl} can be employed; using IPyhon\cite{ipython} in
pylab\cite{matplotlib} mode.

1) Extract keypoints:
In [1]: import fabio 
In [2]: img1 = fabio.open('image1.edf').data
In [3]: import sift
In [4]: siftplan = sift.SiftPlan(template=img1, devicetype="GPU")
In [5]: kp1 = siftplan.keypoints(img1)

After having imported the FabIO\cite{fabio} module in 1, a first
4Mpixel absorption image is read in 2. The library is loaded in 3 and the GPU is
initialized in 4 with all memory allocated on the device. 
In 4, the keypoint extraction took 60 ms for a
4 Mpixel image and returned a 261 keypoint vector as a numpy array.

2) Match keypoints between images
In [6]: img2 = fabio.open('image2.edf').data
in [7]: 
siftplan = sift.SiftPlan(template=img11, devicetype="GPU")
kp1 = siftplan.keypoints(img11)
%timeit kp1 = siftplan.keypoints(img11)
kp1
kp1.size
matchplan = sift.MatchPlan(devicetype="GPU")
matchplan.match(kp1,kp1)
m=matchplan.match(kp1,kp1)
%timeitm=matchplan.match(kp1,kp1)
%timeit m=matchplan.match(kp1,kp1)
im2 = fabio.open("Ti_slow_data_0002_0055_0000_norm.edf").data
%timeit kp2 = siftplan.keypoints(img2)
%timeit kp2 = siftplan.keypoints(im2)
%history

3) Align an image on a reference

   

\subsection{Performances}

\subsection{Limits}
While all calculations are performed in single precision floating point which is
compatible with event the eldest graphic cards supported by OpenCL, the memory
consumption has been traded for performances. Moreover, the algorithm is suited
for linear color scales; making it unsuitable for diffraction images.
Neverthless; using the difference of gaussian method on a pyramid like in
SIFT could be of great help in the extraction of keypoints for calibation of
powder diffraction images.


\section{outlook}

Registration of 3-dimentional object would have a huge application, especially
in the field of tomography and medial images; this field of reseach is very
active in applied mathematics and computer vision.


\section{Conclusion}

\ack{Acknowledgements}

Marine Cotte and Barbara Fayard
Claudio Ferrero and Andy G\"otz

\bibliographystyle{iucr}
\bibliography{biblio}
%\referencelist[biblio]



\end{document}


 

