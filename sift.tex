Image registration and alignment algorithm for synchrotron applications


Image alignment has been requested by many synchrotron beamlines for
various techniques
To address this need we developped a parallel version of the SIFT algorithm
for image registration working both on multi-core system and on graphics cards
and interfaced in Python, a free scripting language very popular among
scientists. This high performance computing library complements the toolbox
based on NumPy\cite{numpy}, SciPy\cite{scipy} and skimage\cite{skimage} 
 
1) Scientific applications for image alignment

1.1 CXDI
1.2 Full-Field X-ray absorption spectroscopy (FFXAS) 

2) Scale Invariant feature transform

2.1 Choice of the algorithm
While phase correlation has intensively been used during the development of
FFXAS for image alignment, this algorithm is limited to translation and turned
out to be very sensitive to artifacts, among those: difference of intensity on
the image border, defects on the syntillator or on the camera\ldots 
Those defect could be corrected by some clever and sample specific
pre-processing (i.e. apodisation of the signal on the borders).

The SIFT algorithm \cite{Lowe1999,Lowe2004} which has been developped
for image registration and widely used for panoramic image stitching has beed
adapted from the IPOL\cite{ASIFT} implementation. Image alignment obtained in
FFXAS were similar in qualty to phase corelation with a better robustness to
artefacts. Moreover keypoints in some defective region of the image can be
discarded easily using a mask.


2.2 how it works


3) 
While the phase correlation algorithm has been easily ported on graphics card
thanks to PyCUDA and cuFFT, hence very fast for online data
pre-processing; 
the SIFT algorithm is much more complicated and the implementation available was
single threaded.

3.1 Parallelization of the algorithm

3.2   

Perspectives: 
Work on sift3D
